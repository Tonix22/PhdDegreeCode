\documentclass{article}
\usepackage[utf8]{inputenc}
\usepackage{cite}

\title{Sample LaTeX Document with Bibliography}
\author{Your Name}
\date{\today}

\begin{document}


\maketitle

\section{Justification}

The capability of telemetry to transmit real-time data efficiently in critical systems is significantly challenged within the evolving landscape of Industry 4.0 \cite{5GReady}. 
Telemetry, which serves as the backbone for transmitting crucial information in many industrial and vehicular applications, faces severe limitations when applied to dynamic environments. 
These challenges are particularly evident in mobility-based systems such as Vehicle-to-Vehicle (V2V) communication \cite{V2V}, where devices are often traveling at high speeds.

In such scenarios, the need for reliable, high-speed data transmission becomes important. 
V2V systems, typically may necessitate data rates ranging from 2 to 54 Mbps at a distance of 15 to 1000m, with a maximal vehicular speed of 200 km/h \cite{DopplereffectV2V}.

Autonomous vehicles depend heavily on uninterrupted, 
real-time data exchange to make split-second decisions, and any delays or inaccuracies in data transmission could result in dangerous situations \cite{AutonomousRemoteSensing}
and could result in the complete loss of equipment or even pose a danger to personnel

Doppler effect and latency will cause distortions in the communication system and demand retransmission or reading of corrupted data \cite{FastFading}.
These elements taken together provide a significant obstacle to reachable: quick fading channels, fast mobility, and high-speed data demand.

\bibliographystyle{plain}
\bibliography{references}


\end{document}
